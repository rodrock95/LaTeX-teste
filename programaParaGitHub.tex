\documentclass[12pt, a4paper]{article}

\usepackage[top = 2cm, bottom = 2cm, left = 2.5cm, right = 2.5cm]{geometry} %pacote de margens
\usepackage[T1]{fontenc}
\usepackage[brazilian]{babel} %pacote de idiomas
\usepackage[utf8x]{inputenc}

\usepackage{setspace}
\onehalfspacing

%Aqui vai ser iniciado o documento
\begin{document}

\title{Minicurso - LaTeX}
\author {Rodrigo Camara Barboza \footnote{Graduado em LCN}}
\date {26 de Setembro de 2022}
\maketitle

"Lorem ipsum dolor sit amet, consectetur adipiscing elit, sed do eiusmod tempor incididunt ut labore et dolore magna aliqua. Ut enim ad minim veniam, quis nostrud exercitation ullamco laboris nisi ut aliquip ex ea commodo consequat. Duis aute irure dolor in reprehenderit in voluptate velit esse cillum dolore eu fugiat nulla pariatur. Excepteur sint occaecat cupidatat non proident, sunt in culpa qui officia deserunt mollit anim id est laborum." \newline
opa
"Lorem ipsum dolor sit amet, consectetur adipiscing elit, sed do eiusmod tempor incididunt ut labore et dolore magna aliqua. Ut enim ad minim veniam, quis nostrud exercitation ullamco laboris nisi ut aliquip ex ea commodo consequat. Duis aute irure dolor in reprehenderit in voluptate velit esse cillum dolore eu fugiat nulla pariatur. Excepteur sint occaecat cupidatat non proident, sunt in culpa qui officia deserunt mollit anim id est laborum."


\begin{enumerate}
	\item Olá
	\item Ok, beleza!
\end{enumerate}

\textit {Esta palavra ficará em itálico}

Esta frase tem o seguinte item \footnote{Que irá para o rodapé}

%\include{cap1}


\end{document}